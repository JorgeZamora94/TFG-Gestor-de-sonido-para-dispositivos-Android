\capitulo{6}{Trabajos relacionados}

\section{No molestar (Android)}
"No molestar" es una configuración en los ajustes de los terminales Android que permite evitar la llegada de llamadas y notificaciones al dispositivo. Se podrán restringir números de teléfono y las diferentes notificaciones como eventos, llamadas y mensajes.\cite{nomolestar}

\section{Netroken Volume Control}
Es una aplicación que nos permite controlar los diferentes tipos de volúmenes dentro de nuestro dispositivo Android, por defecto ya hay unos creados y nosotros podremos crear más. Con la versión de pago podremos utilizar la localización de nuestro dispositivo para activar un perfil de sonido, a demás de programar cambios de perfil.\cite{netroken}

\section{Go Bees}
Go Bees es una aplicación móvil Android creada por el compañero David Miguel de nuestra universidad.
La pongo como trabajo relaccionado ya que es un buen ejemplo de como generar una buena aplicación Android con un seguimiento correcto del proyecto, documentación etc. La pongo como trabajo relacionado ya que lo utilizo como material de referencia para realizar este proyecto.\cite{gobees}

\section{Comparativa y opinión}
Comparando la app Netroken con la del proyecto, vemos que la idea es muy similar en ambas, aunque la aplicación de Netroken gane puntos ya que puede usarse como un gestor de volumen convencional, pierde también parte de su encanto al tener micro pagos para el desbloqueo de todas sus funcionalidades como la de la localización GPS para activar diferentes perfiles.
Pensándolo fríamente, las dos aplicaciones se basan en la misma premisa, pero siguen sin ser lo mismo, ya que este TFG se fundamenta en el control automático de los perfiles de sonido, sin darnos la oportunidad de poder cambiar de perfil de sonido manualmente. Esto a su vez le puede quitar valor con respecto a la aplicación Netroken.

En comparación con la aplicación de no molestar de Android, se podría decir que No Molestar es similar en parte a la configuración de los eventos periódicos, pero con el añadido de que nos permite, no solo configurar el volumen, si no la interacción con otros elementos como el filtrado de las llamadas.