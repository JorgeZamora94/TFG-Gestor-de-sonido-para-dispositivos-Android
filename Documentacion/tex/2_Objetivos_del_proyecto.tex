\capitulo{2}{Objetivos del proyecto}

\paragraph{Objetivo principal}


El primero objetivo que se busca es posibilitar la automatización del control del volumen de nuestro terminal móvil con una sencilla aplicación, y que con el mínimo esfuerzo el usuario pueda configurar distintas configuraciones de audio, para luego activarlas con unos eventos específicos, como eventos de calendario o la localización GPS.

\section{Objetivos técnicos del proyecto}


\begin{itemize}
	\item Se estudiarán las diferentes librerías para el control de calendarios, GPS y wifi.
	\item Se estudiará el funcionamiento de las aplicaciones Android así como el ciclo de vida de estas.
	\item El estudio de diferentes formas de serializar datos en dispositivos móviles.
	\item El estudio de las diferentes versiones de Android, así como de las ventajas que nos ofrece cada una.
	\item La generación de documentación con Latex.
\end{itemize}

\section{Objetivos propios de la aplicación}


\begin{itemize}
	\item Poder guardar configuraciones de sonido propias, para el uso de estas dentro de la aplicación.
	\item Poder generar eventos propios por el usuario, ya sean de forma periódica o para momentos concretos.
	\item Poder obtener la red wifi a la que el usuario está conectado con su dispositivo.
	\item Poder leer los eventos de los calendarios de un usuario.
	\item Poder obtener la posición GPS de nuestro dispositivo Android.
	\item Poder configurar ciertas particualridades de los eventos, como distancia de aproximación en el GPS, lectura de la actual red wifi o poder elegir si queremos los eventos de calendario que su búsqueda sea exacta o que contenga una palabra clave.
	\item Poder obtener una rápida ayuda desde la propia aplicación para utilización de esta.
	\item Poder saber cual es el evento activo en cualquier momento.
	\item Universalizar los layouts de la aplicación.
	\item Internacionalización de la aplicación.
\end{itemize}