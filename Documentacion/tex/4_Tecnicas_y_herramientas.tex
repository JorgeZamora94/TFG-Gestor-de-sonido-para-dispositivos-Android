\capitulo{4}{Técnicas y herramientas}

\section{Herramientas utilizadas}
Las herramientas utilizadas durante el proceso del desarrollo del trabajo han sido, el IDE Android Studio para la generación de código de Java, utilizando las librerías de Android, y a su vez el desarrollo de las interfaces gráficas de nuestra aplicación. Para el control del código generado también utilizamos Códacy, un herramienta que nos califica el código de nuestra aplicación.
Para gestionar las versiones de nuestra aplicación hemos utilizado Git-Hub y para complementarlo y gestionar la planificación del proyecto hemos utilizado Zen-Hub.
Para realizar la documentación hemos utilizado \TeX{}Maker y PowerPoint.

\subsection{Android Studio}

Es un entorno de desarrollo para aplicaciones Android que fue desarrollado por Google en 2014. Utiliza software de IntelliJ IDEA y es gratuito gracias a la licencia de Apache 2.0.
Las plataformas que soporta son Windows, macOS y Linux.

\subsection{\TeX{}Maker}

Es un editor de texto gratuito desarrollado por Pascal Brachet que permite el desarrollo de documentos con La\TeX{}.

\subsection{PowerPoint}

Aplicación de Microsoft que nos permite realizar de manera sencilla presentaciones.

\subsection{Codacy}

Es una herramienta en la nube que nos realiza un estudio sobre la calidad de código que está en un repositorio Git, como GitHub. Cada vez que realicemos un commit el aplicativo se encargará de dar un diagnóstico de la calidad de nuestro código así como de indicarnos donde están los problemas de este.

\subsection{GitHub + ZenHub}
GitHub es una plataforma de desarrollo de aplicaciones. Como su propio nombre indica GitHub utiliza Git, que es un software de control de versiones.
Complementado a GitHub encontramos a ZenHub, que es una extensión para Google Chroome que nos permite el gestionar proyectos con metodología Ágil.

\section{Técnicas utilizadas}

Las técnicas utilizadas a la hora de desarrollar el proyecto han sido dos, a la hora de planificar el desarrollo del proyecto hemos utilizado la metodología Ágil, y a la hora de realizar la programación hemos utilizado la metodología de programación orientada a objetos.

\subsection{Metodología Ágil}

La metodología ágil es una metodología de desarrollo software que se basa en el desarrollar los proyectos de una forma iterativa e incremental, donde los requisitos y sus soluciones van cambiando según las necesidades de ese memento.

Llamaremos iteración a la toma de decisiones y al trabajo que se realiza durante un periodo estipulado de tiempo. Dentro de cada iteración veremos una fase de planificación, de análisis de requisitos, diseño, codificación, pruebas y documentación.

En la fase de planificación lo que se planificará el trabajo que se tendrá que desarrollar en esa iteración. En la fase del análisis de requisitos se verá desarollarán las funcionalidades anteriormente planificadas. En la fase de diseño se verá gestionará la forma de abordar dichos requisitos. Las partes de codificación, documentación y pruebas son más parte del equipo de trabajo del proyecto, y serán los encargados del desarrollo dentro del software de las especificaciones anteriores, teniendo que realizar también la documentación del código generado y de las pruebas asociadas a cada parte del código.\cite{agil}

\subsection{Orientación a objetos.}
La programación orientada a objetos es una metodología de programación que se basa en el tratamiento de objetos que son entidades que tienen un estado, y los cuales pueden realizar unos determinados comportamientos.
Los objetos se descompondrán en atributos y métodos, siendo los atributos los encargados de especificar los estados de los objetos, y los métodos que son los encargados de especificar los comportamientos que pueden realizar dichos objetos.

Las principales características de la programación orientada a objetos son la capacidad para abstraer las características esenciales de los objetos, que todos los elementos pueden considerarse pertenecientes a una misma unidad, permite la herencia entre objetos, es decir que un objeto herede métodos y atributos de otra clase padre, permite la recolección de basura, las propiedades de los objetos están protegidas a modificaciones externas, y se permite la descomposición en partes más pequeñas el código generado.\cite{orientacionobjetos}

\tablaSmall{Herramientas y tecnologías utilizadas en cada parte del proyecto}{l c c c c}{herramientasportipodeuso}
{ \multicolumn{1}{l}{Herramientas} & Aplicación & BD & Memoria \\}{ 
Java & X & X &\\
Librerías Android Java & X & &\\
Android Studio & X & X &\\
Realm & & X &\\
Git + Zen-Hub & X & X & X\\
Mik\TeX{} & & & X\\
PowerPoint{} & & & X\\
\TeX{}Maker & & & X\\
Orientación a objetos & X & &\\
Metodología Ágil del desarrollo & X & X & X\\
} 