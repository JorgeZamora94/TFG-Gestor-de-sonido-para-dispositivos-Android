\capitulo{3}{Conceptos teóricos}
A continuación se expondrán los principales conceptos teóricos, los cuáles son la piedra angular del funcionamiento del proyecto los cuales son los servicios, los listener y la propia concurrencia en si.


\section{Servicios Android - Services}
[Un Service es un componente de una aplicación que puede realizar operaciones de larga ejecución en segundo plano y que no proporciona una interfaz de usuario. Otro componente de la aplicación puede iniciar un servicio y continuará ejecutándose en segundo plano aunque el usuario cambie a otra aplicación. Además, un componente puede enlazarse con un servicio para interactuar con él e incluso realizar una comunicación entre procesos (IPC). Por ejemplo, un servicio puede manejar transacciones de red, reproducir música, realizar I/O de archivos o interactuar con un proveedor de contenido, todo en segundo plano.

Un servicio puede adoptar esencialmente dos formas:

Servicio iniciado
Un servicio está iniciado cuando un componente de aplicación (como una actividad) lo inicia llamando a startService(). Una vez iniciado, un servicio puede ejecutarse en segundo plano de manera indefinida, incluso si se destruye el componente que lo inició. Por lo general, un servicio iniciado realiza una sola operación y no devuelve un resultado al emisor. Por ejemplo, puede descargar o cargar un archivo a través de la red. Cuando la operación está terminada, el servicio debe detenerse por sí mismo.
Servicio de enlace
Un servicio es de enlace cuando un componente de la aplicación se vincula a él llamando a bindService(). Un servicio de enlace ofrece una interfaz cliente-servidor que permite que los componentes interactúen con el servicio, envíen solicitudes, obtengan resultados e incluso lo hagan en distintos procesos con la comunicación entre procesos (IPC). Un servicio de enlace se ejecuta solamente mientras otro componente de aplicación está enlazado con él. Se pueden enlazar varios componentes con el servicio a la vez, pero cuando todos ellos se desenlazan, el servicio se destruye.] \cite{servicios}

\section{Listener}

Los listener son clases que se encargan de controlar los eventos que se generan dentro de nuestra aplicación.
Estos listeners podrán manejar los eventos para su posterior procesado, pudiendo capturar datos e información de dichos eventos.

\section{Concurrencia}
La concurrencia es una propiedad que permite al sistema el poder realizar trabajos de manera simultánea. En nuestro caso, la concurrencia se realizará a través de los diferentes hilos de ejecución de nuestra aplicación, como el hilo principal de la aplicación (Main Thread), nuestro observador GPS el cual mirará en todo momento si se hacen cambios en el GPS, y el hilo del observador genérico, el cual es un hilo que se lanza cada x tiempo.

El principal problema que se trató con el tema de la concurrencia fueron el bloqueo del hilo principal por medio del observador genérico, ya que al utilizar un Thread no había control sobre el bloqueo del hilo principal. Como alternativas se utilizaron los AsincThread, hilos asíncronos que nos permitían la ejecución de tareas en segundo plano, aunque al final se decidió la utilización del TimerTask, ya que está preparado para la repetición de acciones programadas con un lapso x de tiempo.