\apendice{Plan de Proyecto Software}

\section{Introducción}

\section{Introducción}

\section{Planificación temporal}
A continuación se expondrán de manera breve las diferentes fases de desarrollo del proyecto:
Cuando se terminen todas se expodrán los enlaces del repositorio.


\section{Estudio de viabilidad}
A continuación se realizará es estudio de viabilidad económica y legal del proyecto realizado.
\subsection{Viabilidad económica}
En este punto contemplaremos la viabilidad del proyecto desde el punto económico.
\subsubsection{Costes de personal}
Contemos con que el proyecto tiene una duración de 5 meses, en los cuales el programador tendrá un sueldo de 1100 euros.

Cálculos para el presupuesto:

\begin{itemize}
	\item Se calcula un presupuesto de 1100 euros mensuales netos
	\item Con todos los gastos asociados (SS, formación, desempleo...) que contribuyen 37.4 por ciento del sueldo bruto nos dará un total de 657.19 euros de gastos
	\item 1100 + 657.19 =  1757.19 euros a pagar mensualmente la empresa
\end{itemize}

Por parte de la empresa al trabajador, la empresa pagará 1757.19 * 5 = 8785.95 euros



\subsubsection{Costes hardware}
El proyecto se podría realizar con cualquier ordenador con una potencia significativa.
Podríamos hacer una estipulación de unos 1.200 euros de coste hardware para realizar el proyecto.

Amortización = Coste/Vida útil x Tiempo utilizado

Amortización = 1200 euros/48 meses = 25 euros

El total coste hardware es de 1200 euros + 25 * 5 euros con un total de 1325 euros.

\subsubsection{Costes software}
Las herramientas que han sido utilizadas para el desarrollo de este proyecto han sido gratuitas, así que se podría decir que el coste software es 0, teniendo en cuenta que el S.O. del ordenador con el que se desarrolle ya está incluido dentro del presupuesto hardware.

\subsubsection{Costes derivados}
Los costes derivados son costes dentro del proyecto que se pagan indirectamente a la hora de realizar el trabajo, como por ejemplo la luz y el Internet.
El gasto de Internet será de unos 35 euros mensuales.
El gasto de luz será de unos 5 euros mensuales.

Costes derivados: 
40 euros/mes x 5 meses = 200 euros

\subsubsection{Coste total del proyecto}

El coste total del proyecto será de:

\begin{itemize}
	\item 8785.95 euros por parte de pagos de la empresa.
	\item 1325 euros por parte del hardware.
	\item 200 euros por parte de gastos adicionales.
\end{itemize}

\textbf{El total de gastos por parte del desarrollo del proyecto serán de 10310.95 euros}

\subsection{Viabilidad legal}
Mejorar, como enfocar este punto.
\subsubsection{Software}
En la actualidad, casi la totalidad del software utilizado es de creación propia, menos una librería importada a través de gradle para el control de los calendarios.
(Ampliar)

\subsubsection{Documentación}
La documentación ha sido desarrollada por mi persona.

\subsubsection{Imágenes}
Todas las imágenes han sido realizadas a por mí ya sean los gráficos o las impresiones de la app.

