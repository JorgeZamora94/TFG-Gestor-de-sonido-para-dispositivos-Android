\apendice{Plan de Proyecto Software}

\section{Introducción}

\section{Planificación temporal}
A continuación se expondrán de manera breve las 8 fases en las que se dividió el proyecto, las cuales se encuentran diferenciadas en las diferentes millestones que encontramos en el repositorio de GitHub del proyecto.

\begin{itemize}
	\item https://github.com/JorgeZamora94/TFG-Gestor-de-sonido-para-dispositivos-Android/milestone/1
	\item https://github.com/JorgeZamora94/TFG-Gestor-de-sonido-para-dispositivos-Android/milestone/2
	\item https://github.com/JorgeZamora94/TFG-Gestor-de-sonido-para-dispositivos-Android/milestone/3
	\item https://github.com/JorgeZamora94/TFG-Gestor-de-sonido-para-dispositivos-Android/milestone/4
	\item https://github.com/JorgeZamora94/TFG-Gestor-de-sonido-para-dispositivos-Android/milestone/5
	\item https://github.com/JorgeZamora94/TFG-Gestor-de-sonido-para-dispositivos-Android/milestone/6
	\item https://github.com/JorgeZamora94/TFG-Gestor-de-sonido-para-dispositivos-Android/milestone/7
	\item https://github.com/JorgeZamora94/TFG-Gestor-de-sonido-para-dispositivos-Android/milestone/8
\end{itemize}



\section{Estudio de viabilidad}
A continuación se realizará es estudio de viabilidad económica y legal del proyecto realizado.
\subsection{Viabilidad económica}
En este punto contemplaremos la viabilidad del proyecto desde el punto económico.
\subsubsection{Costes de personal}
Contemos con que el proyecto tiene una duración de 5 meses, en los cuales el programador tendrá un sueldo de 1100 euros.

Cálculos para el presupuesto:

\begin{itemize}
	\item Se calcula un presupuesto de 1100 euros mensuales netos
	\item Con todos los gastos asociados (SS, formación, desempleo...) que contribuyen 37.4 por ciento del sueldo bruto nos dará un total de 657.19 euros de gastos
	\item 1100 + 657.19 =  1757.19 euros a pagar mensualmente la empresa
\end{itemize}

Por parte de la empresa al trabajador, la empresa pagará 1757.19 * 5 = 8785.95 euros



\subsubsection{Costes hardware}
El proyecto se podría realizar con cualquier ordenador con una potencia significativa.
Podríamos hacer una estipulación de unos 1.200 euros de coste hardware para realizar el proyecto.

Amortización = Coste/Vida útil x Tiempo utilizado

Amortización = 1200 euros/48 meses = 25 euros

El total coste hardware es de 1200 euros + 25 * 5 euros con un total de 1325 euros.

\subsubsection{Costes software}
Las herramientas que han sido utilizadas para el desarrollo de este proyecto han sido gratuitas, así que se podría decir que el coste software es 0, teniendo en cuenta que el S.O. del ordenador con el que se desarrolle ya está incluido dentro del presupuesto hardware.

\subsubsection{Costes derivados}
Los costes derivados son costes dentro del proyecto que se pagan indirectamente a la hora de realizar el trabajo, como por ejemplo la luz y el Internet.
El gasto de Internet será de unos 35 euros mensuales.
El gasto de luz será de unos 5 euros mensuales.

Costes derivados: 
40 euros/mes x 5 meses = 200 euros

\subsubsection{Coste total del proyecto}

El coste total del proyecto será de:

\begin{itemize}
	\item 8785.95 euros por parte de pagos de la empresa.
	\item 1325 euros por parte del hardware.
	\item 200 euros por parte de gastos adicionales.
\end{itemize}

\textbf{El total de gastos por parte del desarrollo del proyecto serán de 10310.95 euros}

\subsection{Viabilidad legal}
Mejorar, como enfocar este punto.

\subsubsection{Código}
En la actualidad, casi la totalidad del software utilizado es de creación propia, menos una librería importada a través de gradle para su uso, los iconos de la aplicación y un método que contiene el algoritmo para el cálculo de las distancias entre dos puntos sacado de la página de Stack Overflow.

Para el uso de estos contenidos dentro de nuestra aplicación se han tenido que poner sus respectivas referencias dentro de nuestra aplicación (Readme).

También para evitar conflictos de licencias, se ha utilizado la misma licencia que la que utiliza el proveedor de calendarios, la licencia de Apache, en concreto la 2.0.

\subsubsection{Documentación}
Con respecto a la documentación en principio toda ella entera es de autoría propia así como la documentación, anexos, imágenes, vídeos y presentación Power Point. Toda esta documentación utilizará una licencia Creative Commons.

Para incluir referencias de terceros dentro de nuestra docuemtación utilizaremos la bibliografía de esta, donde encontraremos la información de dichas fuentes.

