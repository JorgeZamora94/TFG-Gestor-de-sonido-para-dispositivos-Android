\apendice{Especificación de diseño}

\section{Introducción}
En este apartado veremos cómo hemos estructurado el código y a su vez las estructuras utilizadas para el funcionamiento del proyecto.
\section{Diseño de datos}
Los datos con los que hemos estado trabajando han sido objetos, con lo cual los principales objetos estructurales con los que hemos estado trabajando han sido:
\begin{itemize}
	\item Configuración de sonido.
	\item Evento manual.
	\item Evento periódico.
	\item Evento GPS.
	\item Evento de calendario.
	\item Evento Wifi.
	\item Configuración de la app
\end{itemize}

Con estas clases se ha configurado toda la aplicación para que pueda cambiar de configuración de sonido según la especificación de requisitos del trabajo.



\section{Diseño procedimental}

Faltan las imágenes que se añadirán cuando este

\section{Diseño arquitectónico}

Como diseño arquitectónico podríamos decir que es una aplicación que se basa en el modelo vista controlador, ya que un controlador irá sacando los datos que se requieran a la vista, de un modelo de datos que no está en contacto directo con la vista, siendo el controlador el que haga el paso de la parte front-end a la back-end.


