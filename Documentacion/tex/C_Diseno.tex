\apendice{Especificación de diseño}

\section{Introducción}
En este apartado veremos cómo hemos estructurado el código y a su vez las estructuras utilizadas para el funcionamiento del proyecto.
\section{Diseño de datos}
Los datos con los que hemos estado trabajando han sido objetos, con lo cual los principales objetos estructurales con los que hemos estado trabajando han sido:
\begin{itemize}
	\item Configuración de sonido.
	\item Evento manual.
	\item Evento periódico.
	\item Evento GPS.
	\item Evento de calendario.
	\item Evento Wifi.
	\item Configuración de la app
\end{itemize}

\imagen{relacional}{Imagen que muestra la relación entre las entidades en la base de datos.}

\imagen{clases}{Imagen que muestra la relación entre clases en la aplicación.}

\section{Diseño procedimental}
A continuación se añadirán los diferentes diagramas de secuencia, representando las diferentes acciones que se dan dentro de la aplicación.

\imagen{perfilesdesonido}{Imagen que muestra el diagrama de secuencia a la hora de generar un perfil de sonido.}

\imagen{gps}{Imagen que muestra el diagrama de secuencia a la hora de generar un evento GPS.}

\imagen{manual}{Imagen que muestra el diagrama de secuencia a la hora de generar un evento manual.}

\imagen{periodico}{Imagen que muestra el diagrama de secuencia a la hora de generar un evento periódico.}

\imagen{calendario}{Imagen que muestra el diagrama de secuencia a la hora de generar un evento de calendario.}

\imagen{wifi}{Imagen que muestra el diagrama de secuencia a la hora de generar un evento wifi.}

\imagen{borraevento}{Imagen que muestra el diagrama de secuencia para el borrado de eventos.}

\imagen{timer}{Imagen que muestra el diagrama de secuencia de como funciona el hilo que controla los eventos manuales, periódicos, wifi y calendario.}

\imagen{locationlistener}{Imagen que muestra el funcionamiento del listener GPS.}


\section{Diseño arquitectónico}

Como diseño arquitectónico podríamos decir que es una aplicación que se basa en el modelo vista controlador, ya que un controlador irá sacando los datos que se requieran a la vista, de un modelo de datos que no está en contacto directo con la vista, siendo el controlador el que haga el paso de la parte front-end a la back-end.

\imagen{arquitectura}{Imagen que muestra la arquitectura MVC.}

\subsection{Directorios}
A continuación se añadirán los esquemas de cómo está montada la aplicación a nivel de directorios.

\imagen{directorios}{Imagen que muestra la distribución de los directorios dentro de nuestra aplicación.}

\imagen{dactivities}{Imagen que muestra la distribución de clases en el directorio activities.}

\imagen{dapp}{Imagen que muestra la distribución de clases en el directorio app.}

\imagen{dobservergps}{Imagen que muestra la distribución de clases en el directorio observergps.}

\imagen{dgenerico}{Imagen que muestra la distribución de clases en el directorio observadorgenerico.}

\imagen{dmodificador}{Imagen que muestra la distribución de clases en el directorio modificadorsonido.}

\imagen{dhelp}{Imagen que muestra la distribución de clases en el directorio helpfragments.}

\imagen{dfragment}{Imagen que muestra la distribución de clases en el directorio fragments.}

\imagen{dmodel}{Imagen que muestra la distribución de clases en el directorio bd.model.}

