\capitulo{1}{Introducción}


En los días en los que vivimos, mas de cinco mil millones de personas usan en su vida diaria un SmarthPhone, con lo que a más de una le ha pasado y le pasará, que su terminal suene en situaciones indebidas o que este no suene cuando se quería recibir cualquier tipo de aviso. Lo que quiero desarrollar en este Trabajo de Fin de Grado es una aplicación para SmartPhone, más concreta mente para los que usan el Sistema Operativo Android, para evitar en lo más posible, el que el terminal suene en momentos indebidos, y con ello ahorrarnos alguna que otra vergüenza.


El principal problema que encontramos, es que llevamos encima los terminales, ya sea en el trabajo, durante el estudio, durante las horas de dormir ect. Lo que quiero conseguir es posibilitar el configurar una serie de condiciones para que el terminal cambie solo de estado de volumen, los cuales añadiremos nosotros.


Para facilitar el uso a los usuarios, podremos utilizar los eventos de los calendarios del móvil, así como nuestra situación GPS, por las conexiones Wifi, y por eventos que añadamos manualmente dentro de la aplicación.


Los elementos que podremos configurar serán el volumen general del móvil, el volumen de la música, de la alarma, de las notificaciones y el volumen de llamada.
