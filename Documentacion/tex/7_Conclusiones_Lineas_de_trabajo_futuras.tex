\capitulo{7}{Conclusiones y Líneas de trabajo futuras}

\section{Conclusiones}
Comentándolo desde la parte del desarrollo de la aplicación, Android me ha parecido una tecnología bastante interesante, y a su vez me ha resultado conocida, ya que al trabajar con xml para el desarrollo de interfaces gráficas y la utilización de Java como lenguaje de programación, me ha resultado más sencillo de aprender.

Por otra parte los objetivos que tenía con este proyecto los he cumplido satisfactoriamente, ya que he conseguido crear una aplicación Android, que automatice el cambio de volumen del móvil con los eventos anteriormente citados.

Como opinión sobre la documentación de Android, creo que está bastante bien, aunque sin duda alguna, lo que me ha ayudado en gran medida a desarrollar el proyecto ha sido un curso de Android que he estado realizando como preparación.

Como pega al desarrollo de aplicaciones en Android, no solo necesitarás que en tu emulador funcione, si no que para cada versión y dispositivo. Con lo cual hay que estar siempre actualizando y mejorando tus aplicaciones, o si no se quedarán obsoletas.

\section{Nuevas líneas de trabajo}
A continuación hablaremos de posibles y propuestas a tener en cuenta para seguir con la evolución de este proyecto y de su coste en tiempo y del valor añadido que daría dicha acción.

\subsection{Servidor Web}
Para conseguir una mayor eficacia a la hora de gestionar las configuraciones de sonido de una persona, podríamos tener un servidor web, el cual se encargara de almacenar los datos, permitiéndonos conectarnos con una cuenta.
Coste: 80 - 100.
Valor añadido: 50 -100;

\subsection{Desarrollo para IOS}
Se podría desarrollar la aplicación para que se pueda integrar a los sistemas Apple.
Coste: 100 - 100.
Valor añadido: 100 - 100;

\subsection{Testeo de la aplicación en diferentes dispositivos y versiones}
Para asegurar un correcto funcionamiento de la aplicación, en un futuro se deberán de realizar más pruebas automáticas para luego estas ser probadas en servicios de virtualización de dispositivos, a si de esta manera podremos encontrar posibles errores de funcionamiento.
Coste 50 - 100.
Valor añadido 80 - 100.

\subsection{Publicación de la App}
Se podría publicar la App en el Play Store para que cualquier persona pudiera disfrutar de esta aplicación.
Coste: 10 - 100.
Valor añadido: 100 - 100.

\subsection{Creación de un nuevo evento de "caminando - corriendo"}
Se podría implementar, mediante la utilización del giroscopio del dispositivo, un evento que fuera lanzado cuando el usuario está caminando o corriendo con el dispositivo y habiendo trascurrido un tiempo x desde el inicio del desplazamiento.
Coste: 60 - 100.
Valor añadido: 40 - 100.

\subsection{Añadir control sobre la vibración}
Normalmente el control del volumen siempre va asociado al control de la vibración, con lo que se podría añadir el control de esta en nuestros perfiles de sonido.
Coste: 10 - 100.
Valor añadido: 60 - 100.