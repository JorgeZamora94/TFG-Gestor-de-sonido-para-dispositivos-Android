\capitulo{7}{Conclusiones y Líneas de trabajo futuras}

\section{Conclusiones}
Comentándolo desde la parte del desarrollo de la aplicación, Android me ha parecido una tecnología bastante interesante, y a su vez me ha resultado conocida, ya que al trabajar con xml para el desarrollo de interfaces gráficas y la utilización de Java como lenguaje de programación, me ha resultado más sencillo de aprender.

Por otra parte los objetivos que tenía con este proyecto los he cumplido satisfactoriamente, ya que he conseguido crear una aplicación Android, que automatice el cambio de volumen del móvil con los eventos anteriormente citados.

Como opinión sobre la documentación de Android, creo que está bastante bien, aunque sin duda alguna, lo que me ha permitido desarrollar el proyecto ha sido un curso de Android que he estado realizando como preparación.

Como pega al desarrollo de aplicaciones en Android, no solo necesitarás que en tu emulador funcione, si no que para cada versión y dispositivo.

\section{Nuevas líneas de trabajo}

A continuación hablaremos de posibles y propuestas a tener en cuenta para seguir con la evolución de este proyecto.

\subsection{Servidor Web}
Para conseguir una mayor eficacia a la hora de gestionar las configuraciones de sonido de una persona, podríamos tener un servidor web, el cual se encargara de almacenar los datos, permitiéndonos conectarnos con una cuenta.

\subsection{Conectarse al servidor de fichado del trabajo}

Se podría desarrollar la aplicación para que se conectara al servidor donde se guardan los datos de fichado de los trabajadores, para que cuando entren al trabajo cambien de volumen, y cuando salgan se restablezca la configuración.

\subsection{Desarrollo para IOS}
Se podría desarrollar la aplicación para que se pueda integrar a los sistemas Apple.

\subsection{Desarrollo con diferente paleta de colores, modo noche, daltónicos etc}
Se podría permitir elegir diferentes temas para visualizar la aplicación.

Se realizará una gráfica con relación tiempo de desarrollo y valor aportado a la aplicación de las posibles mejoras y una descripción de cada una de estas, así se podrá ver cuales serían las mejoras a desarrollar con según que previsiones.