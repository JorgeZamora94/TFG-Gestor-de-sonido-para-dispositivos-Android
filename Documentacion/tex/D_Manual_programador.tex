\apendice{Documentación técnica de programación}

\section{Introducción}
A continuación veremos como se ha estado trabajando para generar la aplicación y algunos conceptos necesarios.
\section{Estructura de directorios}

\begin{itemize}
	\item Directorio raíz aplicación: /Codigo/
	\item Directorio de la aplicación: /Codigo/app/
	\item Directorio del código fuente: /Codigo/app/src/
	\item Directorio principal de la app: /Codigo/app/src/main/
	\item Directorio test: /Codigo/app/src/test/
	\item Directorio test Android: /Codigo/app/src/androidTest
	\item Directorio código de la aplicación: /Codigo/app/src/java/
	\item Directorio de los recursos: /Codigo/app/src/res/
	\item Directorio de la documentación: /Documentacion/
	\item Directorio de los textos de la documentación: /Documentacion/text/
	\item Directorio de las imágenes de la documentacion: /Documentacion/img/
\end{itemize}

\section{Manual del programador}
\subsection{Android Studio}
Android Studio\cite{astudio} será el IDE de desarrollo que se utilizará para programar en java, el cual se podrá instalar de una manera rápida y sencilla, permitiéndonos el desarrollo de forma sencilla gracias a las ayudas que este tiene para programar, a sí como las ventajas de poder disponer de un emulador para probar al instante el código que se va generando.
Otra ventaja de utilizar Android Studio, es que cuando se instala este ya nos indica que se instalará la versión de java correspondiente para el desarrollo de las aplicaciones.

\subsection{Ficheros de configuración}
En nuestro proyecto encontraremos tres ficheros de configuración, dos ficheros que utiliza Gradle, para añadir dependencias a nuestro proyecto y lineas de ejecución de este, y un archivo manifest donde indicaremos todo lo relacionado con la ejecución de librerías de android, versionado de la aplicación y diferentes componentes android.

\subsection{Añadir permisos a la aplicación}
Para añadir nuevos permisos a la aplicación, estos tendrán que ser añadidos al archivo AndroidManifest.xml. Aun con todo eso, es posible que dichos permisos también tengan que ser controlados por código.

\subsection{Añadir nuevas librerías/dependencias}
Como se realizó para añadir el Easy Content Provider\cite{easycontent}, para poder añadir una nueva librería, esta se tendrá que añadir al fichero build.gradle(Module:app) y asegurarnos de que se han actualizado las dependencias del proyecto para poder usar la librería con tranquilidad.

\subsection{Cómo saber que version de Android utiliza un proyecto}
Para poder saber qué version de Android se está contemplando, se tendrá que ir al archivo bild.gradle(Module:app) y fijarnos cual es la version de compilación, la versión mínima permitida y cual es la máxima.

\subsection{Importar un nuevo proyecto}
Para importar un nuevo proyecto desde Android Studio simplemente tendremos que ir a /File/Open... y seleccionar el directorio donde se encuentra nuestro proyecto.

\subsection{Comandos básicos de Git}
A continuación se expondrán unos comandos básicos para la utilización de Git \cite{git}.
\begin{itemize}
	\item git push origin rama: todo lo comiteado se sube al repositorio en la nube en la rama establecida.
	\item git commit -m "": realizamos el commit con el mensaje que introduzcamos dentro de las comillas.
	\item git add fichero: añadimos a lo traqueado el fichero.
	\item git add -A .: todo lo que haya en el directorio y sus sub directorios con cambios será añadido.
	\item git clone url: clonamos en la carpeta actual el repositorio señalado en al url.
\end{itemize}

\subsection{Comandos básicos de Git}
Para obtener una revisión de la calidad de nuestro código, podremos utilizar la herramienta Codacy\cite{codacy}, la cual nos sacará una serie de informes para estar al tanto de la calidad de nuestro código, sugiriendo para cada caso cual sería la correcta manera de solventar el problema.

\section{Compilación, instalación y ejecución del proyecto}
Una vez ya realizados los pasos anteriores vistos en el manual del programador podremos realizar los siguiente pasos.
\subsection{Compilación de la aplicación}
Para ello tendremos que tener el proyecto abierto con el Android Studio y seleccionar en la barra de herramientas el botón del desplegable de "Build", y ahí seleccionaremos "Build Project" o "Rebuid Project" si ya lo habíamos buildeado anterior mente.
\subsection{Instalación de la APP y ejecución del proyecto}
Para instalar la aplicación dentro de un simulador Android de IDE Android Studio lo que haremos será crear un dispositivo virtual con la versión de Android que queramos simular. Una vez tengamos creada la máquina virtual de Android lo que haremos será pinchar en el botón de "Run" en la barra de herramientas de Android Studio, o abriendo el desplegable "Run" y seleccionando la opción "Run 'app'". Una vez hecho esto, tendremos que seleccionar un dispositivo sobre el que instalar y ejecutar la aplicación, por consiguiente se nos abrirá una ventana donde seleccionar el dispositivo simulado sobre el que correr la aplicación, permitiéndonos también crear nuevos dispositivos simulados. Una vez seleccionado el dispositivo pincharemos en el botón "OK" y esperaremos a que el proyecto se buildee, y que la máquina virtual se arranque. Una vez arrancada la máquina virtual automáticamente se instalará y se ejecutará.



\section{Pruebas del sistema}
A continuación se expondrán las pruebas que se le han realizado al sistema.

\subsection{Creación de pruebas Espresso}
Para generar pruebas automáticas, estas se podrán generar con el IDE Android Studio, para ello tendremos que ir a /Run/RecordEspressoTest. Una vez seleccionado, se nos abrirá el emulador de Android, en donde cada acción que se realice en la pantalla, esta será guardada, para posteriormente generar un fichero con las rutinas del test.

\subsection{Pruebas unitarias de la base de datos}

\subsubsection{Creación de objetos en la base de datos}

\paragraph{Premisas}
Se tenga instanciada la base de datos y creado el modelo de datos.
Se tenga definida la clase que controle los índices de la base de datos, y esta sea arrancada cuando se inicie la aplicación.

\paragraph{Acciones}
Se generarán todos los tipos de eventos en la base de datos.
Cada vez que se genere una nueva entrada en la base de datos se comprobará que se introduce comprobando que el numero máximo del id generado es mayor al id máximo que había anteriormente.

\paragraph{Resultados esperados}

Se comprueba que se insertan correctamente todos los tipos de eventos en la base de datos.

\paragraph{Resultado}
Prueba superada.

\subsubsection{Obtención de datos de la base de datos}

\paragraph{Premisas}
Se tenga instanciada la base de datos y creado el modelo de datos.
Se tenga definida la clase que controle los índices de la base de datos, y esta sea arrancada cuando se inicie la aplicación.
Haya datos guardados dentro de la base de datos.

\paragraph{Acciones}
Se hará una búsqueda de cada tipo de datos instanciados.
Se comprobará que se puede iterar sobre dichos datos.

\paragraph{Resultados esperados}

Afirmamos que se puede iterar sobre objetos creados de la base de datos y su contenido corresponde con el introducido.

\paragraph{Resultado}
Prueba superada.

\subsection{Pruebas unitarias del observador GPS}

\subsubsection{Obtención de la localización GPS en cualquier momento}

\paragraph{Premisas}
Se tienen que tener aceptados los permisos de localización GPS en la aplicación.


\paragraph{Acciones}
Se generarán todos los tipos de eventos en la base de datos.
Cada vez que se genere una nueva entrada en la base de datos se comprobará que se introduce comprobando que el numero máximo del id generado es mayor al id máximo que había anteriormente.

\paragraph{Resultados esperados}

Se comprueba que se insertan correctamente todos los tipos de eventos en la base de datos.

\paragraph{Resultado}
Prueba superada.

\subsubsection{Acceder a los eventos de calendario}

\paragraph{Premisas}
Se tendrán los permisos de lectura y escritura de calendarios en la aplicación.


\paragraph{Acciones}
Se obtendrán los calendarios y sus eventos para iterar sobre ellos.

\paragraph{Resultados esperados}
Se esperará un listado de eventos sobre el que poder iterar.

\paragraph{Resultado}
Prueba superada.


\subsubsection{Obtener SSID del wifi}

\paragraph{Premisas}
No se requieren premisas.

\paragraph{Acciones}
Se obtendrá la ssid del wifi actual.

\paragraph{Resultados esperados}
Se obtendrá unknown ssid ya que no se puede conectar el emulador a redes wifi.

\paragraph{Resultado}
Prueba superada.

\subsubsection{Observador genérico}

\paragraph{Premisas}
No se requieren premisas.

\paragraph{Acciones}
Se tendrán que crear eventos para ver si la aplicación los detecta.

\paragraph{Resultados esperados}
Si se encuentran eventos que coincidan con las condiciones, se tendrá que informar de ellos.

\paragraph{Resultado}
Prueba superada.

\subsubsection{Borrado de eventos}

\paragraph{Premisas}
Tendrá que haber al menos un evento creado.

\paragraph{Acciones}
Se tendrá que seleccionar un evento de la lista, y seleccionar el botón de borrado de eventos.

\paragraph{Resultados esperados}
El evento seleccionado será borrado.

\paragraph{Resultado}
Prueba superada.

\subsubsection{Bloqueo de eventos}

\paragraph{Premisas}
No hay premisas

\paragraph{Acciones}
Se tendrá se seleccionar la configuración de la app del menú desplegable y guardar la configuración.

\paragraph{Resultados esperados}
Se guarda la configuración de la app, y los eventos no se saltan las restricciones que esta les pone.

\paragraph{Resultado}
Prueba superada.

\subsubsection{Navegación por la ayuda}

\paragraph{Premisas}
No hay premisas.

\paragraph{Acciones}
Navegaremos por las diferentes pestañas de ayuda y por los enlaces de contacto.

\paragraph{Resultados esperados}
Se puede navegar de manera correcta a través de nuestra aplicación y a su vez podremos acceder a las diferentes páginas / aplicaciones de contacto.

\paragraph{Resultado}
Prueba superada.

\subsection{Pruebas automáticas de la aplicación}
Se han realizado una serie de pruebas automáticas con la herramienta de Espresso, para el testeo de aplicaciones Android. Las pruebas realizadas son similares a las anteriormente indicadas, pero estas son realizadas de manera automática por Android Studio. Para ejecutar dichas 