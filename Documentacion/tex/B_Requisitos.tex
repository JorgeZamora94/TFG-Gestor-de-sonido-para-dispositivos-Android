\apendice{Especificación de Requisitos}

\section{Introducción}
En este apartado hablaremos de cuáles son las capacidades del usuario a la hora de trabajar con nuestra aplicación.

\section{Objetivos generales}
De manera general los objetivos de la aplicación son posibilitar la gestión del volumen del terminal de una manera sencilla programando eventos manuales, periódicos, por localización GPS y Wifi, y mediante la lectura de eventos de calendario.

\section{Catalogo de requisitos}
\subsection{Requisitos funcionales}
\begin{enumerate}
	\item El usuario podrá guardar configuraciones de audio.
	\item El usuario podrá guardar eventos periódicos según el día de la semana, seleccionando la hora de inicio y la hora de finalización del evento, y la configuración de audio asociada.
	\item El usuario podrá establecer eventos manuales seleccionando el día del evento, la hora de inicio y de finalización, y la configuración de sonido deseada.
	\item El usuario podrá establecer eventos GPS seleccionando la configuración de audio deseada, la localización GPS será la actual del dispositivo.
	\item El usuario podrá establecer eventos Wifi introduciendo el nombre de la red Wifi deseada, que es a la que luego estará conectado el usuario, y seleccionando una configuración de audio.
	\item El usuario podrá establecer eventos de calendario introduciendo el título del evento del calendario y seleccionando la configuración de audio deseada a cambiar.

\end{enumerate}

\subsection{Requisitos no funcionales}
\begin{enumerate}
	\item La aplicación será de utilización sencilla.
	\item Los menús serán intuitivos.
	\item Encontraremos información básica de funcionamiento dentro de la misma aplicación.
	\item Encontraremos al menos dos idiomas en la aplicación.
\end{enumerate}

\section{Especificación de requisitos}

\subsection{Casos de uso}

\tablaSmall{Crear nueva configuración de audio}{l c c c c}{configuracionaudio}
{ \multicolumn{1}{l}{Información} & Descripción\\}{ 
Versión & V1.0\\
Autores & Jorge Zamora Marqués\\
Requisitos asociados & \\
Descripción & Se creará una nueva configuración de audio.\\
Pre condición & \\
Secuencia normal 
	& Selecciona en el menú la opción de configuraciones de sonido.\\
	
	& Introduce nombre de la configuración\\
	
	& Selecciona volumen del sistema.\\
	
	& Selecciona volumen de la alarma.\\
	
	& Selecciona volumen de la música.\\
	
	& Pincha botón de guardado.\\
	
	& Se guardan los datos en la base de datos.\\
Pos condición & Configuración de audio guardada.\\
Frecuencia esperada & Media\\
Importancia & Alta\\
} 

\tablaSmall{Crear nuevo evento periódico}{l c c c c}{eventoperiodico}
{ \multicolumn{1}{l}{Información} & Descripción\\}{ 
Versión & V1.0\\
Autores & Jorge Zamora Marqués\\
Requisitos asociados & \\
Descripción & Se creará un nuevo evento periódico.\\
Pre condición & Haber configuraciones de audio previamente guardadas\\
Secuencia normal 
	& Selecciona en el menú la opción de eventos periódicos.\\
	
	& Introduce nombre del evento\\
	
	& Selecciona hora de inicio\\
	
	& Selecciona hora de finalización.\\
	
	& Selecciona configuración de audio.\\
		
	& Pincha botón de guardado.\\
	
	& Se guardan los datos en la base de datos.\\
Pos condición & Evento periódico guardado.\\
Frecuencia esperada & Media\\
Importancia & Alta\\
}

\tablaSmall{Crear nuevo evento manual}{l c c c c}{eventomanual}
{ \multicolumn{1}{l}{Información} & Descripción\\}{ 
Versión & V1.0\\
Autores & Jorge Zamora Marqués\\
Requisitos asociados & \\
Descripción & Se creará un nuevo evento manual.\\
Pre condición & Haber configuraciones de audio previamente guardadas\\
Secuencia normal 
	& Selecciona en el menú la opción de eventos manuales.\\
	
	& Introduce nombre del evento\\
	
	& Selecciona día\\
	
	& Selecciona hora de inicio\\
	
	& Selecciona hora de finalización.\\
	
	& Selecciona configuración de audio.\\
		
	& Pincha botón de guardado.\\
	
	& Se guardan los datos en la base de datos.\\
Pos condición & Evento manual guardado.\\
Frecuencia esperada & Media\\
Importancia & Alta\\
} 

\tablaSmall{Crear nuevo evento wifi}{l c c c c}{eventowifi}
{ \multicolumn{1}{l}{Información} & Descripción\\}{ 
Versión & V1.0\\
Autores & Jorge Zamora Marqués\\
Requisitos asociados & \\
Descripción & Se creará un nuevo evento wifi.\\
Pre condición & Haber configuraciones de audio previamente guardadas\\
Secuencia normal 
	& Selecciona en el menú la opción de eventos wifi.\\
	
	& Introduce nombre del evento\\
	
	& Introducimos nombre del wifi.\\
	
	& Selecciona configuración de audio.\\
		
	& Pincha botón de guardado.\\
	
	& Se guardan los datos en la base de datos.\\
Pos condición & Evento wifi guardado.\\
Frecuencia esperada & Media\\
Importancia & Alta\\
} 

\tablaSmall{Crear nuevo evento calendario}{l c c c c}{eventocalendario}
{ \multicolumn{1}{l}{Información} & Descripción\\}{ 
Versión & V1.0\\
Autores & Jorge Zamora Marqués\\
Requisitos asociados & \\
Descripción & Se creará un nuevo evento calendario.\\
Pre condición & Haber configuraciones de audio previamente guardadas\\
Secuencia normal 
	& Selecciona en el menú la opción de eventos de calendario.\\
	
	& Introduce nombre del evento\\
	
	& Introducimos nombre del evento del calendario.\\
	
	& Selecciona configuración de audio.\\
		
	& Pincha botón de guardado.\\
	
	& Se guardan los datos en la base de datos.\\
Pos condición & Evento calendario guardado.\\
Frecuencia esperada & Media\\
Importancia & Alta\\
} 

\tablaSmall{Crear nuevo evento GPS}{l c c c c}{eventogps}
{ \multicolumn{1}{l}{Información} & Descripción\\}{ 
Versión & V1.0\\
Autores & Jorge Zamora Marqués\\
Requisitos asociados & \\
Descripción & Se creará un nuevo evento GPS.\\
Pre condición & Haber configuraciones de audio previamente guardadas\\
Secuencia normal 
	& Selecciona en el menú la opción de eventos GPS.\\
	
	& Introduce nombre del evento\\
	
	& Selecciona configuración de audio.\\
		
	& Pincha botón de guardado.\\
	
	& Se guardan los datos en la base de datos.\\
Pos condición & Evento GPS guardado.\\
Frecuencia esperada & Media\\
Importancia & Alta\\
} 

\tablaSmall{Configurar app}{l c c c c}{configApp}
{ \multicolumn{1}{l}{Información} & Descripción\\}{ 
Versión & V1.0\\
Autores & Jorge Zamora Marqués\\
Requisitos asociados & \\
Descripción & Se cambiará la configuración de los eventos que tratará la app.\\
Pre condición & \\
Secuencia normal 
	& Selecciona en el menú la opción de configuración de la app.\\
	
	& Seleccionamos si queremos tener o no activo cada evento\\
		
	& Pincha botón de guardado.\\
	
	& Se guardan los datos en la base de datos.\\
Pos condición & Configuración guardada.\\
Frecuencia esperada & Baja\\
Importancia & Alta\\
} 