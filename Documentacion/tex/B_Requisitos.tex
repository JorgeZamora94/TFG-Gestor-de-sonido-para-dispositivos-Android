\apendice{Especificación de Requisitos}

\section{Introducción}
En este apartado hablaremos de cuáles son las capacidades del usuario a la hora de trabajar con nuestra aplicación.

\section{Objetivos generales}
De manera general los objetivos de la aplicación son posibilitar la gestión del volumen del terminal de una manera sencilla programando eventos manuales, periódicos, por localización GPS y Wifi, y mediante la lectura de eventos de calendario.

\section{Catalogo de requisitos}
\subsection{Requisitos funcionales}
\begin{enumerate}
	\item El usuario podrá guardar perfiles de audio.
	\item El usuario podrá guardar eventos periódicos introduciendo el nombre del evento, seleccionando el día de la semana, seleccionando la hora de inicio y la hora de fin del evento, y el perfil de audio asociada.
	\item El usuario podrá establecer eventos manuales introduciendo el nombre del evento, seleccionando el día del evento, la hora de inicio y de fin, y seleccionando el perfil de sonido deseada.
	\item El usuario podrá establecer eventos GPS introduciendo el nombre del evento, introduciendo la distancia de reconocimiento, seleccionando el perfil de audio deseada, la localización GPS actual será la que se almacene como punto de control.
	\item El usuario podrá establecer eventos Wifi introduciendo el nombre del evento, introduciendo el nombre de la red Wifi deseada, y seleccionando un perfil de audio.
	\item El usuario podrá establecer eventos de calendario introduciendo el nombre del evento, introduciendo el título del evento del calendario, y seleccionando el perfil de audio.
	\item El usuario podrá borrar cualquier evento.
	\item El usuario podrá activar o desactivar los diferentes tipos de eventos.
\end{enumerate}

\subsection{Requisitos no funcionales}
\begin{enumerate}
	\item La aplicación será de utilización sencilla.
	\item Los menús serán intuitivos.
	\item Encontraremos información básica de funcionamiento dentro de la misma aplicación.
	\item Estará disponible una versión en inglés y otra en castellano.
	\item Se podrá contactar con el soporte de la aplicación desde esta.
\end{enumerate}

\section{Especificación de requisitos}

\subsection{Diagrama de casos de uso}

\imagen{casosdeuso}{Imagen que muestra el diagrama de casos de uso del proyecto.}

\subsection{Casos de uso}
A continuación se mostrarán los casos de uso que han sido desarrollados en la aplicación.

\tablaSmall{Crear nuevo perfil de audio}{l c c c c}{configuracionaudio}
{ \multicolumn{1}{l}{Información} & Descripción\\}{ 
Versión & V1.0\\
Autores & Jorge Zamora Marqués\\
Requisitos asociados & \\
Descripción & Se creará un nuevo perfil de audio.\\
Pre condición & \\
Secuencia normal 
	& Selecciona en el menú la opción de perfiles de sonido.\\
	
	& Introduce nombre del perfil.\\
	
	& Selecciona volumen del sistema.\\
	
	& Selecciona volumen de la alarma.\\
	
	& Selecciona volumen de la música.\\
	
	& Pincha botón de guardado.\\
	
	& Se guardan los datos en la base de datos.\\
Pos condición & Perfil de audio guardado.\\
Frecuencia esperada & Media\\
Importancia & Alta\\
} 

\tablaSmall{Crear nuevo evento periódico}{l c c c c}{eventoperiodico}
{ \multicolumn{1}{l}{Información} & Descripción\\}{ 
Versión & V1.0\\
Autores & Jorge Zamora Marqués\\
Requisitos asociados & \\
Descripción & Se creará un nuevo evento periódico.\\
Pre condición & Haber perfiles de audio previamente guardados\\
Secuencia normal 
	& Selecciona en el menú la opción de eventos periódicos.\\
	
	& Introduce nombre del evento\\
	
	& Selecciona hora de inicio\\
	
	& Selecciona hora de finalización.\\
	
	& Selecciona perfil de audio.\\
		
	& Pincha botón de guardado.\\
	
	& Se guardan los datos en la base de datos.\\
Pos condición & Evento periódico guardado.\\
Frecuencia esperada & Media\\
Importancia & Alta\\
}

\tablaSmall{Crear nuevo evento manual}{l c c c c}{eventomanual}
{ \multicolumn{1}{l}{Información} & Descripción\\}{ 
Versión & V1.0\\
Autores & Jorge Zamora Marqués\\
Requisitos asociados & \\
Descripción & Se creará un nuevo evento manual.\\
Pre condición & Haber perfiles de audio previamente guardados\\
Secuencia normal 
	& Selecciona en el menú la opción de eventos manuales.\\
	
	& Introduce nombre del evento\\
	
	& Selecciona día\\
	
	& Selecciona hora de inicio\\
	
	& Selecciona hora de finalización.\\
	
	& Selecciona perfil de audio.\\
		
	& Pincha botón de guardado.\\
	
	& Se guardan los datos en la base de datos.\\
Pos condición & Evento manual guardado.\\
Frecuencia esperada & Media\\
Importancia & Alta\\
} 

\tablaSmall{Crear nuevo evento wifi}{l c c c c}{eventowifi}
{ \multicolumn{1}{l}{Información} & Descripción\\}{ 
Versión & V1.0\\
Autores & Jorge Zamora Marqués\\
Requisitos asociados & \\
Descripción & Se creará un nuevo evento wifi.\\
Pre condición & Haber perfiles de audio previamente guardados\\
Secuencia normal 
	& Selecciona en el menú la opción de eventos wifi.\\
	
	& Introduce nombre del evento\\
	
	& Introducimos nombre del wifi.\\
	
	& Selecciona perfil de audio.\\
		
	& Pincha botón de guardado.\\
	
	& Se guardan los datos en la base de datos.\\
Pos condición & Evento wifi guardado.\\
Frecuencia esperada & Media\\
Importancia & Alta\\
} 

\tablaSmall{Crear nuevo evento calendario}{l c c c c}{eventocalendario}
{ \multicolumn{1}{l}{Información} & Descripción\\}{ 
Versión & V1.0\\
Autores & Jorge Zamora Marqués\\
Requisitos asociados & \\
Descripción & Se creará un nuevo evento calendario.\\
Pre condición & Haber perfiles de audio previamente guardados\\
Secuencia normal 
	& Selecciona en el menú la opción de eventos de calendario.\\
	
	& Introduce nombre del evento\\
	
	& Introducimos nombre del evento del calendario.\\
	
	& Selecciona perfil de audio.\\
		
	& Pincha botón de guardado.\\
	
	& Se guardan los datos en la base de datos.\\
Pos condición & Evento calendario guardado.\\
Frecuencia esperada & Media\\
Importancia & Alta\\
} 

\tablaSmall{Crear nuevo evento GPS}{l c c c c}{eventogps}
{ \multicolumn{1}{l}{Información} & Descripción\\}{ 
Versión & V1.0\\
Autores & Jorge Zamora Marqués\\
Requisitos asociados & \\
Descripción & Se creará un nuevo evento GPS.\\
Pre condición & Haber perfiles de audio previamente guardados\\
Secuencia normal 
	& Selecciona en el menú la opción de eventos GPS.\\
	
	& Introduce nombre del evento\\
	
	& Introduce la distancia de reconocimiento.\\	
	
	& Selecciona perfil de audio.\\
		
	& Pincha botón de guardado.\\
	
	& Se guardan los datos en la base de datos.\\
Pos condición & Evento GPS guardado.\\
Frecuencia esperada & Media\\
Importancia & Alta\\
} 

\tablaSmall{Configurar app}{l c c c c}{configApp}
{ \multicolumn{1}{l}{Información} & Descripción\\}{ 
Versión & V1.0\\
Autores & Jorge Zamora Marqués\\
Requisitos asociados & \\
Descripción & Se cambiará la configuración de los eventos que tratará la app.\\
Pre condición & \\
Secuencia normal 
	& Selecciona en el menú la opción de configuración de la app.\\
	
	& Seleccionamos si queremos tener o no activo cada evento\\
		
	& Pincha botón de guardado.\\
	
	& Se guardan los datos en la base de datos.\\
Pos condición & Configuración guardada.\\
Frecuencia esperada & Baja\\
Importancia & Alta\\
}

\tablaSmall{Borrar evento}{l c c c c}{borrarEvento}
{ \multicolumn{1}{l}{Información} & Descripción\\}{ 
Versión & V1.0\\
Autores & Jorge Zamora Marqués\\
Requisitos asociados & \\
Descripción & Se podrá borrar cualquier evento creado en la aplicación.\\
Pre condición & Tendrá que haber eventos creados.\\
Secuencia normal 
	& Selecciona en el menú la opción del listado de eventos.\\
	
	& Seleccionamos el evento que queramos borrar.\\
		
	& Pincha botón de borrado.\\
	
	& Se borrará el evento de la base de datos.\\
Pos condición & Evento borrado.\\
Frecuencia esperada & Media\\
Importancia & Alta\\
}